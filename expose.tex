\documentclass[a4paper, 12pt]{amsart}

\usepackage{latexsym, amsmath, hyperref, url, verbatim, lmodern}
\usepackage[all]{xy}
\usepackage[utf8]{inputenc}
\usepackage[T1]{fontenc}
\newcommand{\8}{\infty}
\newcommand{\Horn}[2]{\Lambda^{#1}_{#2}}
\newcommand{\Simplex}[1]{\boldsymbol{\Delta^{#1}}}
\newcommand{\SSet}[1]{\text{Set}_\boldsymbol{\Delta^{#1}}}

\title{An exposé on the $\8$-category of $\8$-categories}

\author{Marie-Camille Delarue}
\author{Ramkumar Ramachandra}

\textwidth=14.5cm
\oddsidemargin=1cm
\evensidemargin=1cm

\begin{document}
\[\begin{abstract}
  This is an exposé on Chapter III of Lurie's first book~\cite{lurie09}. It is meant to be an aid to studying the chapter. We begin with an introduction of Quasi-categories, and orient the reader towards cartesian fibrations. Then, after a brief section on fibrations of simplicial sets, we delve into the main propositions and theorems on $\text{Cat}_\8$. The work was done as a part of the course Catégories supérieures, taught by Muriel Livernet.
\end{abstract}
\maketitle
\tableofcontents

\section{Introduction}

We start with the definititon of quasi-categories, or what will henceforth referred to as $\8$-categories. $K$ is an $\8$-category if inner horns can be filled ($0 < k < n$):

$$
  \begin{xy}
    \xymatrix{
      \Horn{n}{k}\ar[r]\ar@{^{(}->}[d] & K \\
      \Simplex{n}\ar@{.>}[ur]|\exists &
    }
  \end{xy}
$$

\section{Fibrations of simplicial sets}
We
\subsection{Left fibrations}
\subsection{Straightening and unstraightening, in the unmarked case}
\subsection{Simplicial categories, and their \texorpdfstring{$\8$-categories}{∞-categories}}
\subsection{The Joyal model structure}
\subsection{Cartesian fibrations}

\section{\texorpdfstring{$\text{Cat}_\8$}{Cat∞}}
\subsection{Marked anodyne extensions}
\subsection{The Cartesian model structure}
\subsection{Quillen adjunctions between model structures}
\subsection{Straightening and unstraightening, revisited}
\subsection{Relative nerve}
\subsection{Some applications}

\bibliographystyle{amsalpha}
\begin{thebibliography}{99}
  \bibitem{lurie09}
  \href{https://arxiv.org/abs/math/0608040}{Higher Topos Theory}, \\
  Jacob Lurie, \\
  Annals of Mathematics Studies, Number 170, 2009.
\end{thebibliography}
\]
\end{document}