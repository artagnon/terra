\documentclass{amsart}

\usepackage{latexsym, amsmath, amsthm, hyperref, url, verbatim, lmodern, mathrsfs}
\usepackage[all]{xy}
\usepackage[utf8]{inputenc}
\usepackage[T1]{fontenc}
\newcommand{\8}{\ensuremath{\infty}}
\newcommand{\C}{\ensuremath{\mathscr{C}}}
\newcommand{\D}{\ensuremath{\mathscr{D}}}
\renewcommand{\H}{\ensuremath{\mathcal{H}}}
\newcommand{\Horn}[2]{\ensuremath{\Lambda^{#1}_{#2}}}
\newcommand{\Simplex}[1]{\ensuremath{\boldsymbol{\Delta^{#1}}}}
\newcommand{\SSet}{\ensuremath{\text{Set}_{\boldsymbol{\Delta}}}}
\newcommand{\Catinfdel}{\ensuremath{\text{Cat}^{\boldsymbol{\Delta}}_{\infty}}}
\newcommand{\Delt}{\ensuremath{\boldsymbol{\Delta}}}
\newcommand{\op}[1]{{#1}^{\text{op}}}
\newcommand{\Pt}[1]{\{{#1}\}}
\newcommand{\Set}{\ensuremath{\text{Set}}}
\newcommand{\Gpd}{\ensuremath{\text{Gpd}}}
\newcommand{\Map}{\ensuremath{\text{Map}}}
\newcommand{\Fun}{\ensuremath{\text{Fun}}}
\newcommand{\Hom}{\ensuremath{\text{Hom}}}
\newtheorem{definition}{Definition}
\newtheorem{lemma}{Lemma}
\newtheorem{proposition}{Proposition}
\newtheorem{theorem}{Theorem}
\newtheorem{remark}{Remark}

\title{An exposé on the \8-category of \8-categories}

\author{Marie-Camille Delarue}
\author{Ramkumar Ramachandra}

\textwidth=14.5cm
\oddsidemargin=1cm
\evensidemargin=1cm

\begin{document}
\begin{abstract}
  This is an exposé on Chapter III of Lurie's first book~\cite{lurie09}. It is meant to be an aid to studying the chapter. We begin with an introduction of Quasi-categories, and orient the reader towards cartesian fibrations. Then, after a brief section on fibrations of simplicial sets, we delve into the main propositions and theorems on $\text{Cat}_\8$. The work was done as a part of the course Catégories supérieures, taught by Muriel Livernet.
\end{abstract}
\maketitle
\tableofcontents

\section{Introduction}

We start with the definititon of quasi-categories, or what will henceforth referred to as $\8$-categories. $K$ is an $\8$-category if inner horns can be filled ($0 < k < n$):

\begin{definition}[Simplicial nerve]
  Given a small category \C a simplex category $\Delt$, and a map $i : \Delt \rightarrow \C$, there exists a functor $N : \C \rightarrow \Set^{\op{\Delt}}$, or $\C \hookrightarrow \SSet$ called the nerve of the category, which is a simplicial set associated to the category. Applying Yoneda lemma to the Yoneda embedding $N$ yields:

  $$
    N(\C)_n = Hom(\Simplex{n}, N(\C))
  $$
\end{definition}

\begin{lemma}
  The nerve can also be expressed as a kind of simplicial set $K$ if inner horns can be filled uniquely ($0 < k < n$):

  $$
    \begin{xy}
      \xymatrix{
      \Horn{n}{k}\ar[r]\ar@{^{(}->}[d] & K \\
      \Simplex{n}\ar@{.>}[ur]|{\exists!} &
      }
    \end{xy}
  $$
\end{lemma}

This simplicial nerve will play a crucial role in studying \8-categories, and quasi-categories are a natural extension of this definition. We will henceforth refer to quasi-categories as \8-categories.

\begin{definition}[\8-category]
  $K$ is an \8-category if inner horns can be filled ($0 < k < n$):

  $$
    \begin{xy}
      \xymatrix{
        \Horn{n}{k}\ar[r]\ar@{^{(}->}[d] & K \\
        \Simplex{n}\ar@{.>}[ur]|\exists &
      }
    \end{xy}
  $$

  Note that we drop the uniqueness condition from the nerve, because we want composition upto homotopy, not upto strict equality or isomorphism.
\end{definition}

We will now lay out some miscellanous definitions to help the reader with sections II and beyond

\begin{definition}[Overcategory]
\end{definition}

\begin{definition}[Colimit of an \8-category]
\end{definition}

\section{Fibrations of simplicial sets}

\subsection{Left fibrations}
Let's start by defining certain classes of fibrations on simplicial sets, and see where that takes us in the \8-categorical world.

\begin{definition}
  A morphism $f:X\rightarrow S$ is a left fibration if $f$ has the right lifting property with respect to the horn inclusions $\Lambda_i^n\subset \Delta^n, 0\leq i < n$.
\end{definition}

Let's detail the Grothendieck construction, that gives us an equivalence between functors $\xi : \D\rightarrow \Gpd$ and the categories cofibered in groupoids over \D (where \Gpd is the category of groupoids).

\begin{definition}
  Let $F:\C\rightarrow \D$ be a functor between categories. We say that \C is cofibered in groupoids over \D
  if the following conditions are satisfied:
  \begin{itemize}
    \item For every object $C \in \C$ and every morphism $\eta: F(C) \rightarrow D$ in \D , there exists a morphism $\tilde{\eta} : C \rightarrow \tilde{D}$ such that $F(\tilde{\eta}) = \eta$.
    \item For every morphism $\eta: C \rightarrow C'$ in \C and every object $C'' \in \C$, the map
          $$
            Hom_\mathcal{C}(C',C'')\rightarrow Hom_\mathcal{C}(C,C'')\times_{Hom_\mathcal{D}(F(C),F(C''))}Hom_\mathcal{D}(F(C'),F(C'')))
          $$
          is bijective.
  \end{itemize}
\end{definition}

Let \D be a small category and suppose we are given a functor $\xi: \D \rightarrow \Gpd$. We can extract a new category $\C_X$ via the classical Grothendieck construction:

\begin{itemize}
  \item the objects of $\C_X$ are pairs $(D, \eta)$, where $D \in \D$ and $\eta$ is an object of the groupoid $\xi(D)$.
  \item a morphism from $(D, \eta)$ to $(D', \eta')$ in $\C_X$ is given by a pair $(f, \alpha)$, where $f : D \rightarrow D'$ is a morphism in \D and $\alpha : \xi(f)(\eta) \simeq \eta'$ is an isomorphism in the groupoid $\xi(D')$.
\end{itemize}

There is a forgetful functor $F:\C_X \rightarrow \D$, which carries an object $(D, \eta) \in \C_X$ to the underlying object $D \in \D$. It is possible to reconstruct $\xi$ from the category $\C_X$ up to equivalence: if $D$ is an object of \D, then the groupoid $\xi(\D)$ is canonically equivalent to the fiber product $\C_X \times_\D \Pt{D}$.

It so happens that this construction provides an equivalence between functors $\xi : \D \rightarrow \Gpd$ and the categories cofibered in groupoids over \D. Let's see how the theory of left fibrations is an \8-categorical generalization of cofibered groupoids.

\begin{proposition}
  Let $F:\C\rightarrow\D$ be a functor between categories. Then \C is cofibered in groupoids over \D  if and only if the induced map $N(F):N(\C)\rightarrow N(\D)$ is a left fibration of simplicial sets.
\end{proposition}

Let's consider the structure of a general left fibration $p:X\rightarrow S$. When $S$ has a single vertex, $p$ is a left fibration if and only if $X$ is a Kan complex. By stability under pullback, we deduce that for any left fibration $p:X\rightarrow S$ and any vertex $s$ of $S$, the fiber $X_s = X\times_S\{s\}$ is a Kan complex. More precisely, suppose that $f:s\rightarrow s'$ is an edge of the simplicial set $S$
and consider the inclusion $i:X_s\simeq X_s\times\{0\} \subset X_s\times \Delta^1$. We can prove that $i$ is anodyne, and therefore solve the lifting problem:

$$
  \begin{xy}
    \xymatrix{
      \{0\}\times X_s\ar[r]\ar[d] & X \ar[d] &\\
      \Delta^1\times X_s\ar[r]\ar[ur] & \Delta^1\ar[r] & S
    }
  \end{xy}
$$

Restricting the dotted arrow to $\{1\}\times X_s$, we obtain a map $f_!:X_s\rightarrow X_{s'}$, which is uniquely determined up to homotopy.


\begin{lemma}
  Let $q : X \rightarrow S$ be a left fibration of simplicial sets. The assignment $s \in S_0 \mapsto X_s$, $f \in S_1 \mapsto f_!$ determines a functor from the homotopy category $\text{hS}$ into the homotopy category \H of spaces.
\end{lemma}

Therefore, if we take $q: X\rightarrow S$ to be a left fibration of simplicial sets, we can assign to each $\textit{object}$ of $S$ a Kan complex $X_s$, as well as a map $f_{!}: X_s \rightarrow X_{s'}$ for each edge $s \rightarrow s'$. Higher dimensional simplexes of $S$ provide coherence data for these morphisms. A left fibration therefore gives us something like a functor from a simplicial set into the \8-category of spaces. Given a simplicial set $S$, we'd like to specify the relationship between left fibrations over $S$ and functors from $S$ into spaces by exhibiting left fibrations over $S$ as the class of fibrant objects in a particular model category. This category will turn out to be $(\SSet)_{/S}$.

\begin{definition}
  Let $f:X\rightarrow S$ be a map of simplicial sets. The left cone of $f$ $C^
      {\triangleleft}f$ is the simplicial set $\displaystyle S\cup_X X^{\triangleleft} $. Dually, we can define the right cone of $f$ $C^{\triangleright}f:=\displaystyle S\cup_X X^{\triangleright}$.
\end{definition}

We can identify $S$ with a subset of $C^{\triangleleft}f$. With this identification, there is only one vertex of $C^{\triangleleft}f$ that isn't in $S$. We call it the cone point of $C^{\triangleleft}f$.

\begin{definition}[Model structure]. Let $S$ be a simplicial set. A map $f:X\rightarrow Y$ in $(\SSet)_{/S}$ is a
  \begin{itemize}
    \item covariant cofibration if it is a monomorphism of simplicial sets.
    \item covariant equivalence if the induced map \[X^{\triangleleft}\displaystyle \cup_X S \rightarrow Y^{\triangleleft}\cup_Y S\] is a categorical equivalence.
    \item covariant fibration if it has the right lifting property with respect to every map which is an acyclic covariant cofibration.
  \end{itemize}
  These maps determine a left proper combinatorial model structure on $(\SSet)_{/S}$. Combined with the natural simplicial structure, this gives us a simplicial model category structure on $(\SSet)_{/S}$.
\end{definition}

This model structure depends functorially on $S$:
\begin{proposition}
  Let $j:S\rightarrow S'$ be a map of simplicial sets. Let $J_!: (\SSet)_{/S}\rightarrow (\SSet)_{/S'}$ be the forgetful functor and let $j^*$ be its right adjoint given by $j^*X' = X'\times_{S'} S$. Then we have a Quillen adjunction 
  \[(\SSet)_{/S}\underset{j^*}{\overset{j_!}{\overset{\rightarrow}{\leftarrow}}}.\]
\end{proposition}

This model structure has nice properties: for instance, every left anodyne map in $(\SSet)_{/S}$ is a covariant equivalence, every left anodyne map is a trivial cofibration, and most importantly: every covariant fibration is a left fibration of simplicial sets, and every fibrant object of $(\SSet)_{/S}$ determines a left fibration $X\rightarrow S$.

\subsection{Kan fibrations}
Kan fibrations are a subclass of left fibrations. They turn out to be exactly the ones for which the edge maps $f_!:X_s\rightarrow X_{s'}$ are isomorphisms in the category of spaces.

Let $p:X\rightarrow Y$ be a left fibration of simplicial spaces. Then $p$ is a Kan fibration if and only if for every edge $s\rightarrow s'$ in $S$, the map $f_!:X_s\rightarrow X_{s'}$ is an isomorphism in the category of spaces.
We can add a few remarks about specific instances where left fibrations are Kan fibrations:
\begin{itemize}
  \item Any left fibration mapping into a Kan complex is automatically a Kan fibration.
  \item Any left fibration $p:X\rightarrow Y$ such that for each vertex $y\in Y$, the fiber $X_y$ is contractible is a trivial Kan fibration.
\end{itemize}


\subsection{Straightening and unstraightening, in the unmarked case}

Let's look more precisely at the way we can view a left fibration $X\rightarrow S$ as a functor from $S$ into an $\8$-category of Kan complexes. \\
Let us fix a simplicial set $S$, a simplicial category $\mathcal{C}$, and a functor $\phi:\mathscr{C}[S]\rightarrow \mathcal{C}^{op}$. Consider an object $X\in(\SSet)_{/S}$, and let $v$ be the cone point of $X^{\triangleright}$. We can view the simplicial category
\[\mathcal{M} = \mathscr{C}[X^{\triangleright}]\displaystyle \cup_{\mathscr{C}[X]}\mathcal{C}^{op}\] as a correspondence from $\mathcal{C}^{op}$ to $\{v\}$, which we can identify with a simple functor 
\[(St_\phi X)(C)=\text{Map}_\mathcal{M}(C,v).\]
We can consider $St_\phi$ as a functor from $(\SSet)_{/S}$ to $(\SSet)^\mathcal{C}$. We call it the straightening functor associated to $\phi$. 
The straightening functor has a right adjoint, the unstraightening functor, denoted by $Un_\phi$ which satisfies functoriality properties.

\begin{theorem}
  Let $S$ be a simplicial set, $\mathcal{C}$ a simplicial category, and $\phi:\mathscr{C}[S]\rightarrow\mathcal{C}^{op}$ a simplicial functor. The straightening and unstraightening functors determine a Quillen adjunction 
  \[(\SSet)_{/S}\dashv \SSet^\mathcal{C}\]
  where
\end{theorem}


\subsection{Simplicial categories, and their \texorpdfstring{$\8$-categories}{∞-categories}}
\subsection{The Joyal model structure}
\subsection{Cartesian fibrations}

\section{\texorpdfstring{$\text{Cat}_\8$}{Cat∞}}

\begin{definition}
  The simplicial category \Catinfdel is defined as follows:
  \begin{itemize}
    \item The objects are small \8-categories.
    \item Given two \8-categories \C and \D, we define $\Map_{\Catinfdel}(\C, \D)$ to be the largest Kan complex contained in the \8-cateogry $\Fun(\C, \D)$.
  \end{itemize}
\end{definition}

\subsection{Marked anodyne extensions}

\subsection{The Cartesian model structure}
Let $S$ be a simplicial set. In this section, the goal is to define the Cartesian model structure on the category $(\SSet^+)_{/S}$ of marked simplicial sets over $S$. The ultimate goal is to prove that the fibrant objects of $(\SSet^+)_{/S}$ correspond precisely to cartesian fibrations $X\rightarrow S$ and that they encode contravariant functors from $S$ into the \8-category \Catinfdel.
First of all, the category $\SSet^+$ is cartesian-closed. This means that for any two objects $X,Y\in \SSet^+$, there exists an internal mapping object $Y^X$ equipped with an evaluation map $Y^X\times X\rightarrow Y$ which induces bijections 
\[\text{Hom}_{\SSet^+}(Z,Y^X)\rightarrow \text{Hom}_{\SSet^+}(Z\times X,Y)\]
for every $Z\in\SSet^+$. Let $\Map^\flat(X,Y)$ denote the underlying simplicial set of $Y^X$ and $\Map^\sharp(X,Y)$ the simplicial subset of $\Map^\flat(X,Y)$ consisting of all simplices $\sigma\subset \Map^\flat(X,Y)$ such that every edge of $\sigma$ is a marked edge of $Y^X$.\\
These simplicia sets can also be described by the properties:
\begin{align*}
  \Hom_{\SSet}(K,\Map^\flat(X,Y))&\simeq \Hom_{\SSet^+}(K^\flat\times X,Y)\\
  \Hom_{\SSet}(K,\Map^\sharp(X,Y)) &\simeq \Hom_{\SSet^+}(K^\sharp\times X,Y).
\end{align*}

If $X$ and $Y$ are objects of $(\SSet)^+_{/S}$, then we let $\Map_S^\sharp(X,Y) \subset \Map^\sharp(X,Y)$ and $\Map_S^\flat(X,Y) \subset \Map^\flat(X,Y)$ denote the simplicial subsets classifying those maps compatible with the projections to $S$. For instance, in the case where $X\in(\SSet^+)_{/S}$ and $P:Y\rightarrow S$ is a cartesian fibration, tehn $\Map^\flat_S(X,Y^\natural)$ is an \8-category and $\Map^\sharp_S(X,Y^\natural)$ is the largest Kan complex contained in $\Map^\flat_S(X,Y^\natural).$

\begin{proposition}
  Let $S$ be a simplicial set. And let $p:X\rightarrow Y$ be a morphism in $(\SSet^+)_{/S}$. $p$ is a cartesian equivalence if it satisfies either of the following equivalent conditions:
  \begin{itemize}
    \item[(1)] For every cartesian fibration $Z\rightarrow S$, the induced map
    \[\Map^\flat_S(Y,Z^\natural)\rightarrow \Map_S^\flat(X,Z^\natural)\] is an equivalence of \8-categories.
    \item[(2)] For every cartesian fibration $Z\rightarrow S$, the induced map \[\Map^\sharp_S(Y,Z^\natural)\rightarrow \Map_S^\sharp(X,Z^\natural)\] is a homotopy equivalence of Kan complexes.
  \end{itemize}
\end{proposition}

Let's take $f:X\rightarrow Y$ to be a morphism in $(\SSet^+)_{/S}$ which is marked anodyne when regarded as a map of marked simplicial sets. Since the smash product of $f$ with any inclusion $A^\flat\subset B^\flat$ is also marked anodyne, we deduce that the map \[\phi:\Map_S^\flat(Y,Z^\natural)\rightarrow \Map_S^\flat(X,Z^\natural)\] is a trivial fibration for every cartesian fibration $Z\rightarrow S$, which makes $f$ a cartesian equivalence.

\begin{definition}
  $X,Y\in(\SSet^+)_{/S}$ for $S$ a simplicial set. We say that a pair of morphisms $f,g:X\rightarrow Y$ are strongly homotopic if there exists a contractible Kan compelx $K$ and a map $rightarrow \Map^\flat_S(X,Y)$ whose image contains both of the vertices $f$ and $g$. If $Y = Z^\natural$, where $Z\rightarrow S$ is a cartesian fibration, then this just means that $f$ and $g$ are equivalent when viewed as objects of the \8-category $\Map_S^\flat(X,Y)$.
\end{definition}

\begin{proposition}
  Let $X\xrightarrow{p} Y\xrightarrow{q} S$ be a diagram of simplicial sets, where both $q$ and $q\circ p$ are cartesian fibrations. The following assertions are equivalent:
  \begin{itemize}
    \item[(1)] the map $p$ induces a cartesian equivalence $X^\natural \rightarrow Y^\natural$ in $(\SSet^+)_{/S}$.
    \item[(2)] there exists a map $r:Y\rightarrow X$ which is a strong homotopy inverse to $p$, in the sense that $p\circ r$ and $r\circ p$ are both strongly homotopic to the identity.
    \item[(3)] the map $p$ induces a categorical quivalence $X_s\rightarrow Y_s$ for each vertex $s$ of $S$.  
  \end{itemize}
\end{proposition}

Let us now define the model structure we want on $(\SSet^+)_{/S}$:
\begin{proposition}
There exists a left proper combinatorial model structure on $(\SSet^+)_{/S}$ which can be described as follows:
\begin{itemize}
  \item[(C)] The cofibrations in $(\SSet^+)_{/S}$ are the morphisms $p:X\rightarrow Y$ in $(\SSet^+)_{/S}$ which are cofibrations when regarded as morphisms of simplicial sets.
  \item[(W)] The weak equivalences in $(\SSet^+)_{/S}$ are the cartesian equivalences.
  \item[(F)] The fibrations in $(\SSet^+)_{/S}$ are the maps which have the right lifting property with respect to every map which is simultaneously a cofibration and a weak equivalence.  
\end{itemize}
\end{proposition}

\begin{remark}
  Let $S$ be a simplicial set. It is important to be aware of the difference between Cartesian fibrations of simplicial sets and fibrations with respect to the Cartesian model structure on $(\SSet^+)_{/S}$. They are related, though: it turns out that the fibrant objects of $(\SSet^+)_{/S}$ are exactly the objects of the form $X^\natural$, where $X\rightarrow S$ is a Cartesian fibration.
\end{remark}

Let's now look at some properties of the cartesian model structure.

First, the cartesian model structure behaves nicely with respect to products. If $S$ and $T$ are simplicial sets and $Z$ is an object of $(\SSet^+)_{/T}$, then the functor\begin{align*}(\SSet^+)_{/S}&\rightarrow (\SSet^+)_{/S\times T}\\
X&\mapsto X\times Z \end{align*}
preserves Cartesian equivalences.\\
This entails that if we have $f:A\rightarrow B$ a cofibration in $(\SSet^+)_{/S}$ and $f':A'\rightarrow B'$ a cofibration in $(\SSet^+)_{/T}$, then the smash product map
\[(A\times B')\displaystyle \cup_{A\times B}(A'\times B)\rightarrow A'\times B'\] is a cofibration in $(\SSet^+)_{/S\times T}$, which is trivial if either $f$ or $g$ is trivial. All of this brings us to the following result:

\begin{proposition}
  Let $S$ be a simplicial set. Let's regard $(\SSet^+)_{/S}$ as a simplicial category with mapping objects given by $\Map^\sharp_S(X,Y)$. Then $(\SSet^+)_{/S}$ is a simplicial model category.
\end{proposition}

Of course, we can define a second simplicial structure on $(\SSet^+)_{/S}$, where the simplicial mapping spaces are given by $\Map^\flat_S(X,Y)$. This simplicial structure is not compatible with the Cartesian model structure: for fixed $X\in(\SSet^+)_{/S}$ the functor $A\mapsto A^\flat\times X$ does not carry weak homotopy equivalences to Cartesian equivalences. But it does carry categorical equivalences to Cartesian equivalences, and consequently $(\SSet^+)_{/S}$ is endowed with the structure of a \SSet-enriched model category, where we regard \SSet as equipped with the Joyal model structure. It's actually closer to the truth to say that $(\SSet^+)_{/S}$ is a model for an \8-bicategory.


\begin{proposition}
  An object $X\in(\SSet^+)_{/S}$ is fibrant with respect to the cartesian model structure if and only if $X\simeq Y^\natural$< where $Y\rightarrow S$ is a Cartesian fibration.
\end{proposition}

\subsection{Quillen adjunctions between model structures}

\subsection{Straightening and unstraightening, revisited}
Let \C be a category and let $\xi:\C^{op}\text{Cat}$ be a functor from \C to the category $\text{Cat}$ of small categories. Via the Grothendieck construction, we can associate to this data a new category $\tilde{\C}$ which may be described as follows:

\begin{itemize}
  \item The objects of $\tilde{C}$ are pairs $(C,\eta)$, where $C\in\C$ and $\eta\in\xi(C)$.
  \item A morphism from $(C,\eta)$ to $(C', \eta')$ is a pair $(f,\alpha)$, where $f:C\rightarrow C'$ is a morphism in the category \C and $\alpha:\eta\rightarrow \xi(f)(\eta')$ is a morphism in the category $\xi(C)$.
\end{itemize}

This construction establishes an equivalence between Cat-valued functors on $\C^{op}$ and categories which are fibered over $\C$. We will try to establish an \8-categorical version of the equivalence described above. We will replace the category \C by a simplicial set $S$, the category Cat by the \8-category $\text{Cat}_{\8}$, and the notion of fibered category with the notion of Cartesian fibration. This is going to give us an equivalence of \8-categories, which arises from a Quillen equivalence of simplicial model categories. \\
On one side, we have the category $(\SSet^+)_{/S}$, equipped with the Cartesian model structure. On the other hand, we have the category of simplicial functors $$\mathscr{C}[S]^{op}\rightarrow \SSet^+$$ equipped with the projective model structure whose underlying \8-category is equivalent to $\text{Fun}(S^{op},\text{Cat}_\8). $ We have the following key result:

\begin{theorem}
  Let $S$ be a simplicial set, \C a simplicial category, and $\phi:\mathscr{C}[S]\rightarrow \C^{op}$ a functor between simplicial categories. Then there exists a pair of adjoint functors with the following properties:
  \[(\SSet^+)_{/S} \adj{St_\phi^+}{Un_\phi^+} (\SSet^+)^\C\]
\end{theorem}

\subsection{Relative nerve}
\subsection{Some applications}

\bibliographystyle{amsalpha}
\begin{thebibliography}{99}
  \bibitem{lurie09}
  \href{https://arxiv.org/abs/math/0608040}{Higher Topos Theory}, \\
  Jacob Lurie, \\
  Annals of Mathematics Studies, Number 170, 2009.
\end{thebibliography}
\end{document}
