\documentclass[12pt]{amsart}

\usepackage{latexsym, amscd, amsfonts, mathrsfs, amsmath, amssymb, amsthm, xy, stmaryrd, amsxtra, hyperref, url, verbatim}
\usepackage[utf8]{inputenc}
\usepackage[T1]{fontenc}
\newcommand{\8}{\infty}

\title{An exposé on the $\8$-category of $\8$-categories}

\author{Marie-Camille Delarue}
\author{Ramkumar Ramachandra}

\textwidth=14.5cm
\oddsidemargin=1cm
\evensidemargin=1cm

\begin{document}
\begin{abstract}
  This is an exposé on Chapter III of Lurie's first book~\cite{lurie09}. It is meant to be an aid to studying the chapter. We begin with an introduction of Quasi-categories, and orient the reader towards cartesian fibrations. Then, after a brief motivation, we delve into properties of $\textrm{Cat}_\8$. The work was done as a part of the course Catégories supérieures, taught by Muriel Livernet.
\end{abstract}
\maketitle
\tableofcontents

\section{Introduction}
We start with a definititon of $\8$-categories.


\bibliographystyle{amsalpha}
\begin{thebibliography}{99}
  \bibitem{lurie09}
  \href{https://arxiv.org/abs/math/0608040}{Higher Topos Theory}, \\
  Jacob Lurie, \\
  Annals of Mathematics Studies, Number 170, 2009.
\end{thebibliography}
\end{document}
