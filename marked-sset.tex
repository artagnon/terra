\documentclass{beamer}[9pt]
\usetheme{Berlin}
\usecolortheme{dove}
\setbeamercovered{transparent}

\usepackage{latexsym, amsmath, amsthm, mathrsfs, hyperref, url, verbatim, lmodern, xcolor}
\usepackage[all]{xy}
\usepackage[utf8]{inputenc}
\usepackage[T1]{fontenc}

\newcommand{\8}{\ensuremath{\infty}}
\newcommand{\C}{\ensuremath{\mathscr{C}}}
\newcommand{\D}{\ensuremath{\mathscr{D}}}
\newcommand{\J}{\ensuremath{\mathscr{J}}}
\renewcommand{\H}{\ensuremath{\mathcal{H}}}
\newcommand{\Horn}[2]{\ensuremath{\Lambda^{#1}_{#2}}}
\newcommand{\Simplex}[1][n]{\ensuremath{\boldsymbol{\Delta^{#1}}}}
\newcommand{\SSet}{\ensuremath{\text{Set}_{\boldsymbol{\Delta}}}}
\newcommand{\Catinfdel}{\ensuremath{\text{Cat}^{\boldsymbol{\Delta}}_{\infty}}}
\newcommand{\Catinf}{\ensuremath{\text{Cat}_{\infty}}}
\newcommand{\Delt}{\ensuremath{\boldsymbol{\Delta}}}
\newcommand{\op}[1]{\ensuremath{{#1}^{\text{op}}}}
\newcommand{\Pt}[1]{\ensuremath{\{{#1}\}}}
\newcommand{\adj}[2]{\ensuremath{\overset{\overset{#1}{ \rightarrow}}{\underset{#2}{\leftarrow}}}}
\newcommand{\Set}{\ensuremath{\text{Set}}}
\newcommand{\Gpd}{\ensuremath{\text{Gpd}}}
\newcommand{\Map}{\ensuremath{\text{Map}}}
\newcommand{\Fun}{\ensuremath{\text{Fun}}}
\newcommand{\Hom}{\ensuremath{\text{Hom}}}

\newtheorem{proposition}{Proposition}
\newtheorem{remark}{Remark}

\AtBeginSection[]
{
\begin{frame}<beamer>{Table of Contents}
\tableofcontents[currentsection,currentsubsection,
    hideothersubsections,
    sectionstyle=show/shaded,
]
\end{frame}
}

\title{\texorpdfstring{\Catinf}{Cat∞} and Marked Simplicial Sets}
\author{Ramkumar Ramachandra}
\date{16th April 2021}
\institute{Université de Paris}

\begin{document}
\begin{frame}
  \titlepage
\end{frame}

\section{\texorpdfstring{\Catinf}{Cat∞}}

\begin{frame}{The definition of \texorpdfstring{\Catinf}{Cat∞}}
  The simplicial category \Catinfdel is defined as follows:
  \begin{itemize}
    \item The objects are small \8-categories.
    \item Given two \8-categories \C and \D, we define $\Map_{\Catinfdel}(\C, \D)$ to be the largest Kan complex contained in the \8-cateogry $\Fun(\C, \D)$.
  \end{itemize}

  \Catinf denotes the simplicial nerve $\text{N}(\Catinfdel)$, and is referred to as the \8-category of small \8-categories whose mapping spaces are Kan complexes, and composition is strictly associative.
\end{frame}

\section{\texorpdfstring{$\SSet^+$}{Marked simplicial sets}}

\begin{frame}{Motivation}
  The Joyal model structure on \SSet is not compatible with the usual simplicial structure. We will introduce marked simplicial sets $\SSet^+$ as a remedy to the problem, so that we obtain an equivalance of simplicial categories $\Catinfdel \simeq (\SSet^+)^\circ$.
\end{frame}

\begin{frame}{Definition}
  A marked simplicial set is a pair $(X, \epsilon)$, where $X$ is a simplicial set and $\epsilon$ is a set of edges of $X$, which contains every degenerate edge. We will say that $X$ is marked if it belongs to $\epsilon$. A morphism $(X, \epsilon) \rightarrow (X', \epsilon')$ is a map $f : X \rightarrow X'$ having the property that $f(\epsilon) \subseteq \epsilon'$.
\end{frame}

\begin{frame}{$S^\sharp$ and $S^\flat$}
  The two extreme cases of marked simplicial sets are:
  \begin{itemize}
    \item<1-> $S^\sharp = (S, S_1)$ denotes the marked simplicial set in which every edge of $S$ is marked.
    \item<2-> $S^\flat = (S, s_0(S_0))$ denotes the marked simplicial set in which only the degerate edges of $S$ are marked
  \end{itemize}

  \onslide<3>{We let $(\SSet^+)_{/S}$ denote the category which might otherwise be denoted as $(\SSet^+)_{/S^\sharp}$.

    We will soon introduce the cartesian model structure on $(\SSet^+)_{/S}$, and in particular see that each $(\SSet^+)_{/S}$ is a simplicial model category whose fibrant objects are the Cartesian fibrations $X \rightarrow S$.}
\end{frame}

\section{Marked Anodyne Morphisms}

\begin{frame}{Definition}
  The class of marked anodyne morphisms in $\SSet^+$ is the smallest weakly saturated class of morphisms with the following properties:

  \begin{itemize}
    \item[(1)]<1-> For each $0 < i < n$, the inclusion $(\Horn{n}{i})^\flat \subseteq (\Simplex)^\flat$ is marked anodyne.
    \item[(2)]<2> For every $n > 0$ the inclusion
          $$
            (\Horn{n}{n}, \epsilon \cap (\Horn{n}{n})_1) \subseteq (\Simplex, \epsilon)
          $$
          is marked anodyne, where $\epsilon$ denotes the set of all degenerate deges of \Simplex, together with the final edge \Simplex[\{n - 1, n\}].
  \end{itemize}
\end{frame}

\begin{frame}
  \begin{itemize}
    \item[(3)]<1-> The inclusion
          $$
            (\Horn{2}{1})^\sharp \bigsqcup_{(\Horn{2}{1})^\flat} (\Simplex[2])^\flat \rightarrow (\Simplex[2])^\sharp
          $$
          is marked anodyne.
    \item[(4)]<2> For every Kan complex $K$, the map $K^\flat \rightarrow K^\sharp$ is marked anodyne.
  \end{itemize}
\end{frame}

\begin{frame}
  These simplicial sets can also be described by the properties:

  \begin{align*}
    \Hom_{\SSet}(K, \Map^\flat(X, Y))  & \simeq \Hom_{\SSet^+}(K^\flat\times X, Y)  \\
    \Hom_{\SSet}(K, \Map^\sharp(X, Y)) & \simeq \Hom_{\SSet^+}(K^\sharp\times X, Y)
  \end{align*}

  \pause

  If $X$ and $Y$ are objects of $(\SSet)^+_{/S}$, then we let

  \begin{align*}
    \Map_S^\sharp(X, Y) & \subset \Map^\sharp(X, Y) \\
    \Map_S^\flat(X, Y)  & \subset \Map^\flat(X, Y)
  \end{align*}
\end{frame}

\section{Cartesian model structure}

\begin{frame}{Groundwork}
  Let $S$ be a simplicial set. In this section, the goal is to define the Cartesian model structure on the category $(\SSet^+)_{/S}$ of marked simplicial sets over $S$. The ultimate goal is to prove that the fibrant objects of $(\SSet^+)_{/S}$ correspond precisely to cartesian fibrations $X \rightarrow S$ and that they encode contravariant functors from $S$ into the \8-category \Catinfdel.
\end{frame}

\begin{frame}
  First of all, the category $\SSet^+$ is cartesian-closed. This means that for any two objects $X, Y\in \SSet^+$, there exists an internal mapping object $Y^X$ equipped with an evaluation map $Y^X\times X \rightarrow Y$ which induces bijections

  $$
    \Hom_{\SSet^+}(Z, Y^X) \rightarrow \Hom_{\SSet^+}(Z\times X, Y)
  $$

  for every $Z \in \SSet^+$. Let $\Map^\flat(X, Y)$ denote the underlying simplicial set of $Y^X$ and $\Map^\sharp(X, Y)$ the simplicial subset of $\Map^\flat(X, Y)$ consisting of all simplices $\sigma\subset \Map^\flat(X, Y)$ such that every edge of $\sigma$ is a marked edge of $Y^X$.
\end{frame}

\begin{frame}
  These simplicial sets can also be described by the properties:

  \begin{align*}
    \Hom_{\SSet}(K, \Map^\flat(X, Y))  & \simeq \Hom_{\SSet^+}(K^\flat\times X, Y)  \\
    \Hom_{\SSet}(K, \Map^\sharp(X, Y)) & \simeq \Hom_{\SSet^+}(K^\sharp\times X, Y)
  \end{align*}

  \pause

  If $X$ and $Y$ are objects of $(\SSet)^+_{/S}$, then we let

  \begin{align*}
    \Map_S^\sharp(X, Y) & \subset \Map^\sharp(X, Y) \\
    \Map_S^\flat(X, Y)  & \subset \Map^\flat(X, Y)
  \end{align*}

  denote the simplicial subsets classifying those maps compatible with the projections to $S$. For instance, in the case where $X \in (\SSet^+)_{/S}$ and $P: c Y \rightarrow S$ is a cartesian fibration, then $\Map^\flat_S(X, Y^\natural)$ is an \8-category and $\Map^\sharp_S(X, Y^\natural)$ is the largest Kan complex contained in $\Map^\flat_S(X, Y^\natural)$.
\end{frame}

\begin{frame}{Cartesian equivalences}
  Let $S$ be a simplicial set. And let $p: X \rightarrow Y$ be a morphism in $(\SSet^+)_{/S}$. $p$ is a cartesian equivalence if it satisfies either of the following equivalent conditions:

  \begin{itemize}
    \item[(1)] For every cartesian fibration $Z \rightarrow S$, the induced map
          $$
            \Map^\flat_S(Y, Z^\natural) \rightarrow \Map_S^\flat(X, Z^\natural)
          $$
          is an equivalence of \8-categories.
  \end{itemize}
\end{frame}

\begin{frame}
  \begin{itemize}
    \item[(2)] For every cartesian fibration $Z \rightarrow S$, the induced map
          $$
            \Map^\sharp_S(Y, Z^\natural) \rightarrow \Map_S^\sharp(X, Z^\natural)
          $$
          is a homotopy equivalence of Kan complexes.
  \end{itemize}
\end{frame}

\begin{frame}
  Let's take $f: X \rightarrow Y$ to be a morphism in $(\SSet^+)_{/S}$ which is marked anodyne when regarded as a map of marked simplicial sets. Since the smash product of $f$ with any inclusion $A^\flat\subset B^\flat$ is also marked anodyne, we deduce that the map

  $$
    \phi: \Map_S^\flat(Y, Z^\natural) \rightarrow \Map_S^\flat(X, Z^\natural)
  $$

  is a trivial fibration for every cartesian fibration $Z \rightarrow S$, which makes $f$ a cartesian equivalence.
\end{frame}

\begin{frame}{Strong homotopy in $(\SSet^+)_{/S}$}
  $X, Y \in (\SSet^+)_{/S}$ for $S$ a simplicial set. We say that a pair of morphisms $f, g: X \rightarrow Y$ are strongly homotopic if there exists a contractible Kan complex $K$ and a map $rightarrow \Map^\flat_S(X, Y)$ whose image contains both of the vertices $f$ and $g$. If $Y = Z^\natural$, where $Z \rightarrow S$ is a cartesian fibration, then this just means that $f$ and $g$ are equivalent when viewed as objects of the \8-category $\Map_S^\flat(X, Y)$.
\end{frame}

\begin{frame}
  Let $X \xrightarrow{p} Y \xrightarrow{q} S$ be a diagram of simplicial sets, where both $q$ and $q\circ p$ are cartesian fibrations. The following assertions are equivalent:

  \begin{itemize}
    \item[(1)]<1-> the map $p$ induces a cartesian equivalence $X^\natural \rightarrow Y^\natural$ in $(\SSet^+)_{/S}$.
    \item[(2)]<2-> there exists a map $r: Y \rightarrow X$ which is a strong homotopy inverse to $p$, in the sense that $p\circ r$ and $r\circ p$ are both strongly homotopic to the identity.
    \item[(3)]<3> the map $p$ induces a categorical quivalence $X_s \rightarrow Y_s$ for each vertex $s$ of $S$.
  \end{itemize}
\end{frame}

\begin{frame}{Model structure on $(\SSet^+)_{/S}$}
  Let us now define the model structure we want on $(\SSet^+)_{/S}$:

  There exists a left proper combinatorial model structure on $(\SSet^+)_{/S}$ which can be described as follows:
  \begin{itemize}
    \item[(C)]<1-> The cofibrations in $(\SSet^+)_{/S}$ are the morphisms $p: X \rightarrow Y$ in $(\SSet^+)_{/S}$ which are cofibrations when regarded as morphisms of simplicial sets.
    \item[(W)]<2-> The weak equivalences in $(\SSet^+)_{/S}$ are the cartesian equivalences.
    \item[(F)]<3> The fibrations in $(\SSet^+)_{/S}$ are the maps which have the right lifting property with respect to every map which is simultaneously a cofibration and a weak equivalence.
  \end{itemize}
\end{frame}

\begin{frame}
  Let $S$ be a simplicial set. Let's regard $(\SSet^+)_{/S}$ as a simplicial category with mapping objects given by $\Map^\sharp_S(X, Y)$. Then $(\SSet^+)_{/S}$ is a simplicial model category.

  Of course, we can define a second simplicial structure on $(\SSet^+)_{/S}$, where the simplicial mapping spaces are given by $\Map^\flat_S(X, Y)$. This simplicial structure is not compatible with the Cartesian model structure: for fixed $X \in (\SSet^+)_{/S}$ the functor $A\mapsto A^\flat\times X$ does not carry weak homotopy equivalences to Cartesian equivalences. But it does carry categorical equivalences to Cartesian equivalences, and consequently $(\SSet^+)_{/S}$ is endowed with the structure of a \SSet-enriched model category, where we regard \SSet as equipped with the Joyal model structure. It's actually closer to the truth to say that $(\SSet^+)_{/S}$ is a model for an \8-bicategory.
\end{frame}

\section{Comparison of model structures}

\begin{frame}{The different model structures}
  We have a plethora of model structures on categories of simplicial sets over the simplicial set $S$:

  \begin{itemize}
    \item<1->[(0)] Let $\C_0$ denote $(\SSet)_{/S}$ endowed with the Joyal model structure. Cofibrations are monomorphisms of simplicial sets, and weak equivalences are categorical equivalences.
    \item<2>[(1)]  Let $\C_1$ denote $(\SSet^+)_{/S}$ endowed with the marked model structure. Cofibrations are maps $(X, \epsilon_X) \rightarrow (Y, \epsilon_Y)$ which induce monomorphisms $X \rightarrow Y$, and the weak equivalences are Cartesian equivalences.
  \end{itemize}
\end{frame}

\begin{frame}
  \begin{itemize}
    \item<1->[(2)]  Let $\C_2$ denote $(\SSet^+)_{/S}$ endowed with the following localization of the Cartesian model structure: $f : (X, \epsilon_X) \rightarrow (Y, \epsilon_Y)$ is a cofibration if $X \rightarrow Y$ is a monomorphism, and a weak equivalence if $f : X^\sharp \rightarrow Y^\sharp$ is a marked equivalence in $(\SSet^+)_{/S}$.
    \item<2->[(3)]  Let $\C_3$ denote $(\SSet)_{/S}$ endowed with the covariant model structure. The cofibrations are monomorphisms and the weak equivalcnes are contravariant equivalences.
    \item<3>[(4)] Let $\C_4$ denote $(\SSet)_{/S}$ endowed with the usual homotopic-theoretic model structure. The cofibrations are monomorphisms of simplicial sets, and the weak equivalences are weak homotopy equivalences of simplicial sets.
  \end{itemize}
\end{frame}

\begin{frame}{The Quillen adjunctions}
  There exists a sequence of Quillen adjunctions:

  $$
    \C_0 \adj{F_0}{G_0} \C_1 \adj{F_1}{G_1} C_2 \adj{F_2}{G_2} \C_3 \adj{F_3}{G_3} \C_4
  $$
\end{frame}

\begin{frame}{Description of adjunction functors}
  The functors may be described as follows:

  \begin{itemize}
    \item<1->[(A0)] $G_0$ is the forgetful functor from $(\SSet^+)_{/S}$ to $(\SSet)_{/S}$, which igonres the collection of marked edges. $F_0$ is left adjoint to $G_0$, given by $X \mapsto X^\flat$. The Quillen adjunction $(F_0, G_0)$ is a Quillen equivalance if $S$ is a Kan complex.
    \item<2->[(A1)] $F_1$ and $G_1$ are identity functors on $(\SSet^+)_{/S}$.
    \item<3->[(A2)]  $F_2$ is the forgetful functor from $(\SSet^+)_{/S}$ to $(\SSet)_{/S}$ which ignores the collection of marked edges. $G_2$ is right adjoint to $F_2$ given by $X \mapsto X^\sharp$. The Quillen adjunction $(F_2, G_2)$ is a Quillen equivalence for every simplicial set $S$.
    \item<4>[(A3)]  $F_3$ and $G_3$ are identity functors on $(\SSet)_{/S}$. The Quillen adjunction $(F_3, G_3)$ is Quillen equivalence whenever $S$ is a Kan complex.
  \end{itemize}
\end{frame}
\end{document}
